% !TEX root = ./article.tex

\documentclass{article}

\usepackage{mystyle}
\usepackage{myvars}

%-----------------------------

\begin{document}

	\maketitle

%-----------------------------
%	TEXT
%-----------------------------


  \section{Demostración}

    \paragraph{}
    Sea $(X, Y)$ un vector bidimiensional de variables aleatorias. Se presuponen conocidas la función de densidad  $f_X(x)$ (o de probabiblidad $P(X = x)$ en caso de ser discreta) de la variable $X$. También se asume como conocida la distribución de la función de densidad de variable $Y$ condicionada por cualquier valor de $X$, es decir, $f_Y(y \mid X = x)$ (o la de probabilidad $P(Y = y \mid X = x)$ en caso de ser discreta).

    \paragraph{}
    Lo que se pretende obtener a partir de dichas funciones de distribución es la ley de probabilidad que sigue la variable $Y$, es decir, su función de densidad $f_Y(y)$ (o de probabiblidad $P(Y = y)$ en caso de ser discreta).

    \paragraph{}
    Para obtener dicho valor, se hará uso de la \emph{ley de probabibilidades totales}, que indica que si un suceso $A$ que se da sobre un espacio muestral $\omega$ puede particionarse en $n$ partes determinadas por $B=\{B_1, ..., B_i, ..., B_n \}$ y se conoce la distribución de probabilidades tanto de estas ($P(B_i)  \ \forall i \in \{1,...,n\}$), como de las de $A$ condicionada a ellas ($P(A \mid B_i) \ \forall i \in \{1,...,n\}$), entonces se puede conocer la probabilidad del suceso $A$ tal y como se indica en la ecuación \eqref{eq:_law_of_total_probabilities}.

    \begin{align}
    \label{eq:_law_of_total_probabilities}
      P(A) = \sum_{i=1}^nP(A \mid B_i)P(B_i)
    \end{align}

    \paragraph{}
    La idea de la ecuación \eqref{eq:_law_of_total_probabilities} se puede extender al uso de variables, tanto discretas como continuas. Por tanto en las ecuaciones \eqref{eq:marginal_1}, \eqref{eq:marginal_2}, \eqref{eq:marginal_3} y \eqref{eq:marginal_3} para las cuatro posibles combinaciones de variables continuas y discretas.

    \begin{align}
      \text{X e Y son Discretas:}&\\
    \label{eq:marginal_1}
      P(Y=y) &= \sum_iP(Y = y \mid X = x_i)P(X = x_i)\\
      \text{X e Y son Continuas:}&\\
    \label{eq:marginal_2}
      f_{Y}(y) &=\int_{-\infty}^{+\infty}f_{Y}(y | X = x)f_X(x) dx \\
      \text{X es Discreta e Y Continua:}&\\
    \label{eq:marginal_3}
      f_{Y}(y) &= \sum_if_{Y}(y \mid X = x_i)P(X = x_i)\\
      \text{X es Continua e Y Discreta:}&\\
    \label{eq:marginal_4}
      P(Y = y) &=\int_{-\infty}^{+\infty}P( Y =y | X = x)f_X(x) dx \\
    \end{align}
%-----------------------------
%	Bibliographic references
%-----------------------------
	\nocite{prob2017}

  \bibliographystyle{alpha}
  \bibliography{bib}

\end{document}
